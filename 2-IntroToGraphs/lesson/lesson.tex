\documentclass[12 pt, twoside] {article}
\usepackage[margin=1in]{geometry}
\usepackage[utf8]{inputenc}
\usepackage{listings}
\usepackage{color}
\usepackage{textcomp}
\usepackage{setspace}
\usepackage{verbatim}
\usepackage{graphicx}
\usepackage{footnote}
\usepackage{enumitem}
\usepackage{fancyhdr}
\usepackage[parfill]{parskip}

\makesavenoteenv{tabular}
\makesavenoteenv{table}

\setlist[itemize]{noitemsep, topsep=0pt}

\newcommand\la{\textlangle}
\newcommand\ra{\textrangle}

\setlength{\parindent}{0pt}

\definecolor{codegreen}{rgb}{0,0.6,0}
\definecolor{codegray}{rgb}{0.5,0.5,0.5}
\definecolor{codeblue}{rgb}{0,0,0.6}
\definecolor{backcolor}{rgb}{0.95,0.95,0.95}

\lstdefinestyle{mystyle}{
	backgroundcolor = \color{backcolor},
	commentstyle = \color{codeblue},
	keywordstyle = \color{codegreen},
	numberstyle = \color{codegray},
	stringstyle = \color{magenta},
	basicstyle = \footnotesize\ttfamily,
	breakatwhitespace = false,
	breaklines = true,
	captionpos = b,
	keepspaces = true,
	numbers = left,
	numbersep = 5pt,
	showspaces = false,
	showstringspaces = false,
	showtabs = false,
	tabsize = 4
}

\lstset{style = mystyle}

\pagestyle{fancy}
\fancyhf{}
\rhead{Yicheng Wang}
\lhead{Graph Theory 1}

\begin{document}
{\catcode`?=\active
\def?!#1!{\footnote{#1}}
\section*{Introducation to Graph Theory}
\subsection*{Definition and Representation}

In mathematical and computer science terms, a graph is a collection of values
(known as nodes or vertices) and a collection of connections, known as edges.
This is frequently represented by endpoints and line segments connecting them.
Mathematically a graph is a tuple of two sets, $G(E,V)$, an edge set and a
vertex set. Graph functions as a template to abstract many problems, it
represents relationships between similar constructs.

For the purposes of complexity analysis, the runtime is often expressed in terms
of the size of the $E$ and $V$ sets, similar to how complexities of operations
with 1D data structures is often in terms of the length. In most cases, we can
make the assumption of $E \approx V^2$. This is because in a fully connected
graph, the size of the edge set is equal to $V(V-1)$, and in contests it is
often best to assume the worst unless there's a good reason for acting
otherwise.

There are many types of graphs, and each of them are used to represent different
relationships between values, and their difference is mostly on what the edges
can be. To start off there's directed and undirected graphs, in directed graphs
the edges only go in one direction whereas in undirected graphs the edges go in
both. There are further restrictions on whether the edge length is specified, as
well as the allowed range of the length, for example, some graphs call for
negative edge lengths while others restrict the edge lengths to only positive.
For the purposes of this lecture, we are going to deal with directed graphs with
positive edge lengths.

I know that we usually don't go into implementation in this class. But the
implementation of a graph can also help us visualize information. There are
two major ways of representing graph:
\begin{itemize}
    \item Adjacency Matrix: A 2D array of booleans (or edge lengths) where the
        indices are the nodes connecting the edge.
    \item Edge List: This involves a slight modification of the node data
        structure, which is usually just a 1D array, but this modification
        involves adding an additional list of connected vertices.
\end{itemize}

The advantage of using adjacency matrix is that it's much simpler to code up,
the disadvantage is that it uses a lot of space. But I digress, back to graph
algorithms.

\subsection*{Depth First Search and Applications}

The most basic algorithm we will study today is the depth first search,
otherwise known as depth first transversal. The point of this algorithm is to
find a path from one node to another. Let's look through an example together.

We are going to solve a maze. A maze can be represented as a graph where each
grid point represent a vertex and if two points are ``adjacent," i.e. If you can
reach one square from another, then they are connected by an edge in our
representation.

The algorithm goes as follows, we keep a list of booleans corresponding to the
list of vertices. This list marks if a square has been visited or not. We start
by the beginning of the maze.

At each crossroads (aka each vertex with more than one edge coming out of it).
We will recursively search along each of the paths until it reaches either an
exit, in which case we're done, or a dead end, or a visited node. In either of
the later cases we go one the next branch, hence earning the algorithm its name
\textbf{depth first search (DFS)}, because it always going the full depth along
each path.

This algorithm has the complexity of $O(E + V)$, as in the worst case we are
guaranteed to visit every vertex, and in doing so we have to traverse every
edge. Using the earlier substitution, the worse case scenario can be estimated
to be about $O(V^2)$.

\begin{verbatim}
                    +#                                        
                     #################                      
                     #   #        #  #                      
                     #   #        #  ##################     
                     #   #        #            #      #     
                     #   #        #            #      #     
                     #   #######  #   ##########      #     
                     ###       #  #            #      #     
                     ####################      #      #     
                     #####       #                    #     
                                 #       #            #     
                      ################   ###############    
                                    #            #######    
                                    #                       
                                    ##########$             
\end{verbatim}

This algorithm is actually very versatile and has profound implications, for
example, it can be used to detect if a graph is connected or have several
disconnected pieces. This can be done by trying to traverse the graph with no
endpoint, only seeking to mark all nodes as visited. If the call returns and
there still exists unvisited nodes, those nodes are disconnected from the
starting point, allowing us to subdivide an unconnected graph into connected
pieces.

Another modification of this algorithm is cycle detection. Recall the few
terminating conditions, another one we haven't used yet is reaching a previously
visited node. Think about what this means. If we hit a previously visited node,
that means there is a cycle in our graph starting at that node. This will also
come in handy in problems.

\subsection*{Breath First Search and Meet In The Middle}

An alternative method of transversal is known as breath first search or
transversal. As its name suggests, unlike depth first search this method takes
one step in all possible directions. This is a different algorithm but the basic
tenement is the same. As a transversal algorithm, it can do everything depth
first search can, including connectedness and cycle detection. This approach has
two advantages. The first being that if we are looking for a path between two
nodes, breath first search guarantees an optimal solution. The second is that it
has a much better average time complexity than depth first search. Consider we
are trying to get from point A to point B, but A branches into B and C, and in
DFS, if we stumble into C, we will reach the worst case before we get to B, but
with breath first search we are guaranteed to reach B in one step.

In breath first search, if the average degree of the graph is $p$ and the true
distance between the starting and ending positions is $d$, then $O(p^d)$ nodes
will be traversed, whereas DFS is much more erratic and ``luck based.'' In
short, unless traversal of the whole graph is necessary, almost always prefer
BFS to DFS. (But since DFS is easier to implement, it's better to know both!)

\subsubsection*{Meet in the Middle}

Let's talk about an interview question. Given the name of two people on
Facebook, how does one find their degrees of separation.

This is one of those cases where DFS would not solve the problem for rather
obvious reasons. But let's talk about BFS, starting from friend A, if we were to
find the distance to friend B via BFS, we will have to traverse $O(p^d)$ people
as previously stated.

Sociology theory says that $d$ is bounded by 6, and let's put a lower bound on
$p$ as, say, 150. This means that we will need to traverse about $150^6 =
11390625000000$ people. This is not good enough.

Consider, however, that we concurrently conduct BFS from both A and B. Each with
their own unique visited-A and visited-B markings. Then if they encounter the
visited marking of the other person, we are done. This effectively reduced the
complexity of $O(p^d)$ to $O(2 \times p^{d/2})$, which further reduces to
$O(p^{d/2})$, a drastic decrease. In our example it reduces the number of people
traversed to $2 * 150^3 = 6750000$, which is much, much less.

Although we introduced this technique in the context of graph theory, this
technique can be used in a variety of settings. However, this technique is
usually known as the ``smart brute-force'' method, and you should only try to do
it when there is no other algorithms.

Let's see an example outside the context of graphs, and yet another popular
interview question, the four number problem!

The problem is as follows, you are given a list of integers with length $n$,
find if there exists 4 number $a, b, c, d$ s.t. $a + b + c + d = 0$.

The brute force solution is to try all combination of 4 numbers, which is
$O(n^4)$. For those interview oldies, you probably know that ``a hashtable makes
everything better," so a slightly better solution is to hash all possible $-(a +
b + c)$ and check that against the original array.

That last approach is incredibly close to the correct one, but as the name
suggests, meet in the \textbf{middle} requires using symmetry to our maximum
advantage. The insight is that $-(a + b) = (c + d)$. We still use a hashset, and
store all possible $-(a + b)$'s, then we iterate through all possible $(c +
d)$'s and check them against the subset. This has complexity of $O(n^2)$, which
is better than either of the previous ones.
\end{document}
